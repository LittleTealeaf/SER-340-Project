\documentclass{article}
\usepackage{graphicx}
\usepackage[margin=0.75in]{geometry}
\usepackage{enumitem}
\usepackage{titlesec}

% \setcounter{secnumdepth}{4}

\graphicspath{{../../out/docs/srs/diagrams/}}

\setlist{nosep}





\title{Software Requirements Specification}

\author{Thomas Kwashnak, Isaac Crawford, and Sadjell Mamon}


\begin{document}

% \setlist[itemize]{leftmargin=0.5in}

\begin{titlepage}
  \vspace*{1cm}

  \Huge
  \textbf{Software Requirement Specifications}

  \vspace{0.5cm}

  \textbf{for}

  \vspace{0.5cm}

  \textbf{Code Grader}

  \vspace{2cm}

  \LARGE
  \textbf{Version 1.0}

  \vspace{3cm}

  Prepared by

  Thomas Kwashnak, Isaac Crawford, and Sadjell Mamon

  \vspace{0.75cm}
  Quinnipiac University

  \vspace{7cm}

  \today
\end{titlepage}

\section*{Revision History}

\begin{tabular}{| p{0.2\linewidth} | p{0.075\linewidth} | p{0.2\linewidth} | p{0.4\linewidth} |}
  \hline
  Date & Version & Description & Author\\
  \hline
  \hline
  TBD & TBD & TBD & TBD \\
  \hline

\end{tabular}

\newpage

\tableofcontents

\newpage

\section{Introduction}

\subsection{Purpose}
\paragraph{} The purpose of this document is to outline the requirements for the development of the Code Grader System. This document will provide information about the functionality the Code Grader system will have. This project will be completed by Spring 2023

\paragraph{} This project is designed to help professors grade assignments faster and more efficiently. Students will be able to join classes and submit their assignments; while teachers will be able to add students to their classes and see their submissions.

\subsection{Intended Audience and Reading Suggestion}

This document is designed for project managers and developers to help them understand the clients' requirements. It contains a project's description, requirements, and use cases. The reader should start with the description and end with the requirements.

\subsection{References}
<List any other documents or Web addresses to which this SRS refers. These may include user interface style guides, contracts, standards, system requirements specifications, use case documents, or a vision and scope document. Provide enough information so that the reader could access a copy of each reference, including title, author, version number, date, and source or location.>
\section{Overall Description}

\subsection{Product Perspective}
<Describe the context and origin of the product being specified in this SRS. For example, state whether this product is a follow-on member of a product family, a replacement for certain existing systems, or a new, self-contained product. If the SRS defines a component of a larger system, relate the requirements of the larger system to the functionality of this software and identify interfaces between the two. A simple diagram that shows the major components of the overall system, subsystem interconnections, and external interfaces can be helpful.>

\subsection{Product Functions}
<Summarize the major functions the product must perform or must let the user perform. Details will be provided in Section 3, so only a high-level summary (such as a bullet list) is needed here. Organize the functions to make them understandable to any reader of the SRS. Add a high-level architecture diagram here. Describe the diagram and add caption.>

\subsection{User Classes and Characteristics}

\begin{tabular}{| p{0.3\linewidth} | p{0.65\linewidth} |}
  \hline
  \textbf{Class} & \textbf{Description} \\
  \hline
  \hline
  Physicians & \begin{itemize}
    \item Highly experienced and educated in creating and performing medical processes
    \item High-level security priviledges for working with patient data
    \item Little to no experience with the Smart Checklist system
  \end{itemize} \\
  \hline
\end{tabular}


\subsection{Operating Environment}
<Describe the environment in which the software will operate, including the hardware platform, operating system and versions, and any other software components or applications with which it must peacefully coexist.>

\subsection{Design and Implementation Constraints}
<Describe any items or issues that will limit the options available to the developers. These might include: corporate or regulatory policies; hardware limitations (timing requirements, memory requirements); interfaces to other applications; specific technologies, tools, and databases to be used; parallel operations; language requirements; communications protocols; security considerations; design conventions or programming standards (for example, if the customers organization will be responsible for maintaining the delivered software).>

\subsection{Assumptions and Dependencies}
<List any assumed factors (as opposed to known facts) that could affect the requirements stated in the SRS. These could include third-party or commercial components that you plan to use, issues around the development or operating environment, or constraints. The project could be affected if these assumptions are incorrect, are not shared, or change. Also identify any dependencies the project has on external factors, such as software components that you intend to reuse from another project, unless they are already documented elsewhere (for example, in the vision and scope document or the project plan).>


\newpage
\section{Specific Requirements}

\subsection{Interfaces}

\subsection{Requirements}

\begin{itemize}
  \item \textbf{REQ-1} The system should allow an admin to login using their username and password
  \item \textbf{REQ-2} The system should allow an instructor to login using their username and password
  \item \textbf{REQ-3} The system should allow a student to login using their username and password
  \item \textbf{REQ-4} The system should allow an admin to create an instructor account with a username and password
  \item \textbf{REQ-5} The system should allow an instructor to create a new course with a name and description
  \item \textbf{REQ-6} The system should allow an instructor to create a student account for a course with a username and a password
  \item \textbf{REQ-7} The system should allow an instructor to create an assignment
  \item \textbf{REQ-8} The system should allow an instructor to upload a test case to an assignment, including the script input and expected output
  \item \textbf{REQ-9} The system should allow an instructor to update an assignment
  \item aowiejwoaef

\end{itemize}

\newpage

\subsection{Use Case Diagram}

\includegraphics[height=8in,width=\linewidth,keepaspectratio]{Use Case Diagram.png}

\newpage

\subsection{System Features}

\subsubsection{Login}

\paragraph{Description} An admin, instructor, or student signs into the application using their username and password

\paragraph{Functional Requirements and Use Cases}

\begin{itemize}
  \item \textbf{REQ-1} The system should allow an admin to login using their username and password
  \item \textbf{REQ-2} The system should allow an instructor to login using their username and password
  \item \textbf{REQ-3} The system should allow a student to login using their username and password
\end{itemize}


\vspace{0.1in}

\begin{tabular}{| p{0.25\linewidth} | p{0.65\linewidth} |}
  \hline
  \textbf{Use Case Name} & Login \\
  \hline
  \textbf{Brief Description} & The user logs into their account using their username and password \\
  \hline
  \textbf{Actor(s)} & User \\
  \hline
  \textbf{Pre-conditions} & None\\
  \hline
  \textbf{Basic Flow} & \begin{itemize}
    \item[] \textbf{1} The user opens up the web application
    \item[] \textbf{2} The system prompts the user to sign in using a username and password
    \item[] \textbf{3} The user provides their account's username and password, and submits
    \item[] \textbf{4} The system displays the user's home page
  \end{itemize}\\
  \hline
  \textbf{Alternate Flows} & \begin{itemize}
    \item[] \textbf{3A} The user inputs a username that does not exist
    \item[] \textbf{3A1} The system displays an error stating that either the username or password is incorrect, and asks the user to try signing in again
    \item[] \textbf{3B} The user inputs a password that does not match the user's actual password
    \item[] \textbf{3B1} The system displays an error stating that either the username or password is incorrect, and asks the user to try signing in again
  \end{itemize} \\
  \hline
\end{tabular}

\subsubsection{Create Instructor Account}

\paragraph{Description} An admin creates an instructor account, setting it's username and password


\paragraph{Functional Requirements and Use Cases}

\begin{itemize}
  \item \textbf{REQ-4} The system should allow an admin to create an instructor account with a username and password
\end{itemize}

\vspace{0.1in}

\begin{tabular}{| p{0.25\linewidth} | p{0.65\linewidth} |}
  \hline
  \textbf{Use Case Name} & Create Instructor Account \\
  \hline
  \textbf{Brief Description} & The admin creates an instructor account, and specifies the username and password \\
  \hline
  \textbf{Actor(s)} & Admin \\
  \hline
  \textbf{Pre-conditions} & The admin is logged in, and is on the admin homepage\\
  \hline
  \textbf{Basic Flow} & \begin{itemize}
    \item[] \textbf{1} The admin chooses to create a new instructor account
    \item[] \textbf{2} The system prompts the admin to specify the instructor name, username, and password
    \item[] \textbf{3} The admin fills in the name, username and password, and submits
    \item[] \textbf{4} The system creates the account and returns the admin to the admin homepage
  \end{itemize}\\
  \hline
  \textbf{Alternate Flows} & \begin{itemize}
    \item[] \textbf{3A} The user inputs a username that alraedy exists
    \item[] \textbf{3A1} The system displays an error asking the admin to choose another username
  \end{itemize} \\
  \hline
\end{tabular}

\subsubsection{Create Course}

\paragraph{Description} The instructor creates a course with a name and description


\paragraph{Functional Requirements and Use Cases}
\begin{itemize}
  \item \textbf{REQ-5} The system should allow an instructor to create a new course with a name and description
\end{itemize}

\vspace{0.1in}

\begin{tabular}{| p{0.25\linewidth} | p{0.65\linewidth} |}
  \hline
  \textbf{Use Case Name} & Create Course \\
  \hline
  \textbf{Brief Description} & The instructor creates a course \\
  \hline
  \textbf{Actor(s)} & Instructor \\
  \hline
  \textbf{Pre-conditions} & The instructor is logged in, and is currently on the instructor homepage\\
  \hline
  \textbf{Basic Flow} & \begin{itemize}
    \item[] \textbf{1} The instructor chooses to create a new course
    \item[] \textbf{2} The system prompts the instructor to fill in the new course name and description
    \item[] \textbf{3} The instructor fills in the details and submits
    \item[] \textbf{4} The system creates the course, and displays the course's page
  \end{itemize}\\
  \hline
  \textbf{Alternate Flows} & \begin{itemize}
    \item[] \textbf{3A} The instructor already has a course with the given coursename
    \item[] \textbf{3A1} The system displays an error asking the instructor to choose another course name
  \end{itemize} \\
  \hline
\end{tabular}

\subsubsection{Create Student Account}

\paragraph{Description} The instructor creates a student accoutn for a course

\paragraph{Functional Requirements and Use Cases}

\begin{itemize}
  \item \textbf{REQ-6} The system should allow an instructor to create a student account for a course with a username and a password
\end{itemize}

\vspace{0.2in}

\begin{tabular}{| p{0.25\linewidth} | p{0.65\linewidth} |}
  \hline
  \textbf{Use Case Name} & Create Student Account \\
  \hline
  \textbf{Brief Description} & The instructor creates a student account for a course \\
  \hline
  \textbf{Actor(s)} & Instructor \\
  \hline
  \textbf{Pre-conditions} & The instructor is logged in, and is at the home page of one of their coruses\\
  \hline
  \textbf{Basic Flow} & \begin{itemize}
    \item[] \textbf{1} The instructor chooses to view the student list
    \item[] \textbf{2} The system shows the user a list of all currently enrolled students, and the option to create a new account
    \item[] \textbf{3} The instructor chooses to create a new account
    \item[] \textbf{4} The system prompts the instructor to input the new account's name, username, and password
    \item[] \textbf{5} The user inputs the relevant information and submits
    \item[] \textbf{6} The system creates the user, and then shows an updated list of all students currently enrolled in that course.
  \end{itemize}\\
  \hline
  \textbf{Alternate Flows} & \begin{itemize}
    \item[] \textbf{5A} The instructor inputs a username that already has an account
    \item[] \textbf{5A1} The system asks the instructor to input a different username because the username is currently in use.
  \end{itemize} \\
  \hline
\end{tabular}

\subsubsection{Upload Test Case}

\paragraph{Description} An instructor uploads files required to create a test case for an assignment

\paragraph{Functional Requirements and Use Cases}

\begin{itemize}
  \item \textbf{REQ-?} The system should allow an instructor to upload an input and expected output file as a test case.
\end{itemize}

\vspace{0.2in}
\begin{tabular}{| p{0.25\linewidth} | p{0.65\linewidth} |}
  \hline
  \textbf{Use Case Name} & Upload Test Case \\
  \hline
  \textbf{Brief Description} & The instructor uploads a test case as two files including the input and output \\
  \hline
  \textbf{Actor(s)} & Instructor \\
  \hline
  \textbf{Pre-conditions} & The instructor is logged in, and is currently editing or creating an assignment\\
  \hline
  \textbf{Basic Flow} & \begin{itemize}
    \item[] \textbf{1} The instructor chooses to upload a test case
    \item[] \textbf{2} The system prompts the instructor to upload two documents, one labelled `Input' and the other labelled `Output', as well as the option to show detailed results to the students
    \item[] \textbf{3} The instructor uploads the input and output files and submits
    \item[] \textbf{4} The system creates the test case, attaches it to the assignment, and returns the instructor to the assignment editor
  \end{itemize}\\
  \hline
  \textbf{Alternate Flows} & \begin{itemize}
    \item[] \textbf{3A} The instructor does not upload an input file
    \item[] \textbf{3A1} The system prints an error and asks the instructor to upload an input file
    \item[] \textbf{3B} The instructor does not upload an output file
    \item[] \textbf{3B1} The system prints and error and asks the instructor to upload an output file
  \end{itemize} \\
  \hline
\end{tabular}

\subsubsection{Create and Update Assignments}

\paragraph{Description} An instructor is able to create and update assignemnts within a course

\paragraph{Functional Requirements and Use Cases}

\begin{itemize}
  \item \textbf{REQ-?} The system allows an instructor to create an assignment with a name, description, and test cases
  \item \textbf{REQ-?} The system allows an instructor to edit any assignment in one of their courses
\end{itemize}

\vspace{0.2in}

\begin{tabular}{| p{0.25\linewidth} | p{0.65\linewidth} |}
  \hline
  \textbf{Use Case Name} & Create Assignment \\
  \hline
  \textbf{Brief Description} & The instructor creates an assignment in one of their couses \\
  \hline
  \textbf{Actor(s)} & Instructor \\
  \hline
  \textbf{Pre-conditions} & The instructor is logged in\\
  \hline
  \textbf{Basic Flow} & \begin{itemize}
    \item[] \textbf{1} The instructor chooses a course
    \item[] \textbf{2} The system displays a list of assignments in that course
    \item[] \textbf{3} The instructor chooses to createan assignment
    \item[] \textbf{4} The system prompts the instructor to input the assignment name, description, and upload any test cases
    \item[] \textbf{5} The instructor inputs the proper data, and submits the assignment
    \item[] \textbf{6} The system displays an updated assignment list with the new assignment
  \end{itemize}\\
  \hline
  \textbf{Alternate Flows} & \begin{itemize}
    \item[] \textbf{3A} The instructor does not upload an input file
    \item[] \textbf{3A1} The system prints an error and asks the instructor to upload an input file
    \item[] \textbf{3B} The instructor does not upload an output file
    \item[] \textbf{3B1} The system prints and error and asks the instructor to upload an output file
  \end{itemize} \\
  \hline
\end{tabular}

\vspace{0.2in}

\begin{tabular}{| p{0.25\linewidth} | p{0.65\linewidth} |}
  \hline
  \textbf{Use Case Name} & Update Assignment \\
  \hline
  \textbf{Brief Description} & The instructor creates an assignment in one of their couses \\
  \hline
  \textbf{Actor(s)} & Instructor \\
  \hline
  \textbf{Pre-conditions} & The instructor is logged in\\
  \hline
  \textbf{Basic Flow} & \begin{itemize}
    \item[] \textbf{1} The instructor chooses a course
    \item[] \textbf{2} The system displays a list of assignments in that course
    \item[] \textbf{3} The instructor chooses to edit one of the assignments in the course
    \item[] \textbf{4} The system displays an editable screen for the assignment, allowing the instructor to edit the name, description, and upload new test cases
    \item[] \textbf{5} The user updates the assignment and submits
    \item[] \textbf{6} The system updates the assignment and displays an updated assignment list
  \end{itemize}\\
  \hline
  \textbf{Alternate Flows} & None \\
  \hline
\end{tabular}

\subsubsection{Submit Assignments}

\paragraph{Description} A student completes and submits an assignemnt for grading

\paragraph{Functional Requirements and Use Cases}

\begin{itemize}
  \item \textbf{REQ-??} The system should allow a student to upload and submit their code for an assignment
\end{itemize}

\vspace{0.2in}

\begin{tabular}{| p{0.25\linewidth} | p{0.65\linewidth} |}
  \hline
  \textbf{Use Case Name} & Submit Assignment \\
  \hline
  \textbf{Brief Description} & The student uploads their code and submits an assignment \\
  \hline
  \textbf{Actor(s)} & Student \\
  \hline
  \textbf{Pre-conditions} & The Student is Logged In\\
  \hline
  \textbf{Basic Flow} & \begin{itemize}
    \item[] \textbf{1} The student chooses an assignment to complete
    \item[] \textbf{2} The system displays details for that assignment, and gives the student the ability to upload code and submit
    \item[] \textbf{3} The student uploads their code and submits
    \item[] \textbf{4} The system uploads the code and submits the assignment
    \item[] \textbf{5} The system grades the assignment, and displays the grading results to the user
  \end{itemize}\\
  \hline
  \textbf{Alternate Flows} & \begin{itemize}
    \item[] \textbf{3A} The student does not upload any file and submits
    \item[] \textbf{3A1} The system prints an error requesting that the user uploads a code file to submit
  \end{itemize}\\
  \hline
\end{tabular}

\subsubsection{View Student Grades}

\paragraph{Description} The instructor views the submission grades of students in a course

\paragraph{Functional Requirements and Use Cases}

\begin{itemize}
  \item\textbf{REQ-??} The system should allow an instructor to view a list of grades in a course
\end{itemize}

\vspace{0.2in}

\begin{tabular}{| p{0.25\linewidth} | p{0.65\linewidth} |}
  \hline
  \textbf{Use Case Name} & View Course Grades \\
  \hline
  \textbf{Brief Description} & The instructor navigates and views a list of course grades \\
  \hline
  \textbf{Actor(s)} & Instructor \\
  \hline
  \textbf{Pre-conditions} & The Instructor is Logged In\\
  \hline
  \textbf{Basic Flow} & \begin{itemize}
    \item[] \textbf{1} The instructor chooses a course
    \item[] \textbf{2} The system displays the list of assignments for the course
    \item[] \textbf{3} The instructor chooses to view course grades
    \item[] \textbf{4} The system displays a table of students and their grades for each assignment
  \end{itemize}\\
  \hline
  \textbf{Alternate Flows} & None\\
  \hline
\end{tabular}

\subsubsection{Download Student Grades}

\paragraph{Description} The instructor downloads student grades as a file

\paragraph{Functional Requirements and Use Cases}

\begin{itemize}
  \item \textbf{REQ-??} The system should allow an instructor to download course grades as a file
\end{itemize}

\vspace{0.2in}

\begin{tabular}{| p{0.25\linewidth} | p{0.65\linewidth} |}
  \hline
  \textbf{Use Case Name} & Download Course Grades \\
  \hline
  \textbf{Brief Description} & The instructor downloads student grades for a particular course \\
  \hline
  \textbf{Actor(s)} & Instructor \\
  \hline
  \textbf{Pre-conditions} & The Instructor is Logged In and is viewing grades for one of their courses\\
  \hline
  \textbf{Basic Flow} & \begin{itemize}
    \item[] \textbf{1} The instructor chooses to download the course grades
    \item[] \textbf{2} The system asks the instructor what file format they want the grades in
    \item[] \textbf{3} The instructor selects a file format
    \item[] \textbf{4} The system compiles the grades into a file of the provided format, and sends it to the instructor
  \end{itemize}\\
  \hline
  \textbf{Alternate Flows} & None\\
  \hline
\end{tabular}

\subsubsection{Provide Submission Feedback}

\paragraph{Description} The instructor provides feedback to the submission of a particular student for a particular assignment

\paragraph{Functional Requirements and Use Cases}

\begin{itemize}
  \item \textbf{REQ-??} The system should allow the instructor to add feedback to a student's assignment submissions
\end{itemize}

\begin{tabular}{| p{0.25\linewidth} | p{0.65\linewidth} |}
  \hline
  \textbf{Use Case Name} & Add Submission Feedback \\
  \hline
  \textbf{Brief Description} & The instructor provides feedback to a submission \\
  \hline
  \textbf{Actor(s)} & Instructor \\
  \hline
  \textbf{Pre-conditions} & The Instructor is Logged In and is currently viewing the list of assignments for a course\\
  \hline
  \textbf{Basic Flow} & \begin{itemize}
    \item[] \textbf{1} The instructor chooses one of the assignment grades for a student
    \item[] \textbf{2} The system displays that submission details, including the submitted code and the test case results. The system also displays an option to add feedback as a comment
    \item[] \textbf{3} The instructor chooses to add feedback
    \item[] \textbf{4} The system prompts the user to input the feedback
    \item[] \textbf{5} The instructor inputs the feedback and submits
    \item[] \textbf{6} The system adds the feedback to the submission and displays the submission
  \end{itemize}\\
  \hline
  \textbf{Alternate Flows} & None\\
  \hline
\end{tabular}

\newpage
\appendix




\section{Glossary}

\paragraph{Test Case} A test case is an execution instance that tests submission code for expected results. Test cases consist of two parts: an input and output. The input is what is fed into the code, and the output is what the code should output. The actual output is compared with the expected output to pass or fail the test.


\newpage
\section{Analysis Models}


\newpage
\section{To Be Determined List}


\end{document}
