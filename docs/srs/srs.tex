\documentclass{article}
\usepackage{graphicx}
\usepackage[margin=0.75in]{geometry}
\usepackage{enumitem}
\usepackage{titlesec}

% \setcounter{secnumdepth}{4}

\graphicspath{{../../out/docs/srs/diagrams/}}

\setlist{nosep}





\title{Software Requirements Specification}

\author{Thomas Kwashnak, Isaac Crawford, and Sadjell Mamon}


\begin{document}

% \setlist[itemize]{leftmargin=0.5in}

\begin{titlepage}
  \vspace*{1cm}

  \Huge
  \textbf{Software Requirement Specifications}

  \vspace{0.5cm}

  \textbf{for}

  \vspace{0.5cm}

  \textbf{Code Grader}

  \vspace{2cm}

  \LARGE
  \textbf{Version 1.0}

  \vspace{3cm}

  Prepared by

  Thomas Kwashnak, Isaac Crawford, and Sadjell Mamon

  \vspace{0.75cm}
  Quinnipiac University

  \vspace{7cm}

  \today
\end{titlepage}

\section*{Revision History}

\begin{tabular}{| p{0.2\linewidth} | p{0.075\linewidth} | p{0.2\linewidth} | p{0.4\linewidth} |}
  \hline
  Date & Version & Description & Author\\
  \hline
  \hline
  TBD & TBD & TBD & TBD \\
  \hline

\end{tabular}

\newpage

\tableofcontents

\newpage

\section{Introduction}

\subsection{Purpose}
<Briefly identify the product whose software requirements are specified in this document, including the revision or release number. Describe the scope of the product that is covered by this SRS, particularly if this SRS describes only part of the system or a single subsystem.>

\subsection{Intended Audience and Reading Suggestion}

< Briefly describe the different types of reader that the document is intended for, such as developers, project managers, marketing staff, users, testers, and documentation writers. Describe what the rest of this SRS contains and how it is organized. Suggest a sequence for reading the document, beginning with the overview sections and proceeding through the sections that are most pertinent to each reader type.>

\subsection{References}
<List any other documents or Web addresses to which this SRS refers. These may include user interface style guides, contracts, standards, system requirements specifications, use case documents, or a vision and scope document. Provide enough information so that the reader could access a copy of each reference, including title, author, version number, date, and source or location.>
\section{Overall Description}

\subsection{Product Perspective}
<Describe the context and origin of the product being specified in this SRS. For example, state whether this product is a follow-on member of a product family, a replacement for certain existing systems, or a new, self-contained product. If the SRS defines a component of a larger system, relate the requirements of the larger system to the functionality of this software and identify interfaces between the two. A simple diagram that shows the major components of the overall system, subsystem interconnections, and external interfaces can be helpful.>

\subsection{Product Functions}
<Summarize the major functions the product must perform or must let the user perform. Details will be provided in Section 3, so only a high-level summary (such as a bullet list) is needed here. Organize the functions to make them understandable to any reader of the SRS. Add a high-level architecture diagram here. Describe the diagram and add caption.>

\subsection{User Classes and Characteristics}

\begin{tabular}{| p{0.3\linewidth} | p{0.65\linewidth} |}
  \hline
  \textbf{Class} & \textbf{Description} \\
  \hline
  \hline
  Physicians & \begin{itemize}
    \item Highly experienced and educated in creating and performing medical processes
    \item High-level security priviledges for working with patient data
    \item Little to no experience with the Smart Checklist system
  \end{itemize} \\
  \hline
\end{tabular}


\subsection{Operating Environment}
<Describe the environment in which the software will operate, including the hardware platform, operating system and versions, and any other software components or applications with which it must peacefully coexist.>

\subsection{Design and Implementation Constraints}
<Describe any items or issues that will limit the options available to the developers. These might include: corporate or regulatory policies; hardware limitations (timing requirements, memory requirements); interfaces to other applications; specific technologies, tools, and databases to be used; parallel operations; language requirements; communications protocols; security considerations; design conventions or programming standards (for example, if the customers organization will be responsible for maintaining the delivered software).>

\subsection{Assumptions and Dependencies}
<List any assumed factors (as opposed to known facts) that could affect the requirements stated in the SRS. These could include third-party or commercial components that you plan to use, issues around the development or operating environment, or constraints. The project could be affected if these assumptions are incorrect, are not shared, or change. Also identify any dependencies the project has on external factors, such as software components that you intend to reuse from another project, unless they are already documented elsewhere (for example, in the vision and scope document or the project plan).>


\newpage
\section{Specific Requirements}

\subsection{Interfaces}

\subsection{Requirements}

\begin{itemize}
  \item \textbf{REQ-1} The system should allow an admin to login using their username and password
  \item \textbf{REQ-2} The system should allow an instructor to login using their username and password
  \item \textbf{REQ-3} The system should allow a student to login using their username and password
  \item \textbf{REQ-4} The system should allow an admin to create an instructor account with a username and password
  \item \textbf{REQ-5} The system should allow an instructor to create a new course with a name and description

\end{itemize}

\subsection{System Features}

\subsubsection{Login}

\paragraph{Description} An admin, instructor, or student signs into the application using their username and password

\paragraph{Functional Requirements and Use Cases}

\begin{itemize}
  \item \textbf{REQ-1} The system should allow an admin to login using their username and password
  \item \textbf{REQ-2} The system should allow an instructor to login using their username and password
  \item \textbf{REQ-3} The system should allow a student to login using their username and password
\end{itemize}


\vspace{0.1in}

\begin{tabular}{| p{0.25\linewidth} | p{0.65\linewidth} |}
  \hline
  \textbf{Use Case Name} & Login \\
  \hline
  \textbf{Brief Description} & The user logs into their account using their username and password \\
  \hline
  \textbf{Actor(s)} & User \\
  \hline
  \textbf{Pre-conditions} & None\\
  \hline
  \textbf{Basic Flow} & \begin{itemize}
    \item[] \textbf{1} The user opens up the web application
    \item[] \textbf{2} The system prompts the user to sign in using a username and password
    \item[] \textbf{3} The user provides their account's username and password, and submits
    \item[] \textbf{4} The system displays the user's home page
  \end{itemize}\\
  \hline
  \textbf{Alternate Flows} & \begin{itemize}
    \item[] \textbf{3A} The user inputs a username that does not exist
    \item[] \textbf{3A1} The system displays an error stating that either the username or password is incorrect, and asks the user to try signing in again
    \item[] \textbf{3B} The user inputs a password that does not match the user's actual password
    \item[] \textbf{3B1} The system displays an error stating that either the username or password is incorrect, and asks the user to try signing in again
  \end{itemize} \\
  \hline
\end{tabular}

\subsubsection{Create Instructor Account}

\paragraph{Description} An admin creates an instructor account, setting it's username and password


\paragraph{Functional Requirements and Use Cases}

\begin{itemize}
  \item \textbf{REQ-4} The system should allow an admin to create an instructor account with a username and password
\end{itemize}

\vspace{0.1in}

\begin{tabular}{| p{0.25\linewidth} | p{0.65\linewidth} |}
  \hline
  \textbf{Use Case Name} & Create Instructor Account \\
  \hline
  \textbf{Brief Description} & The admin creates an instructor account, and specifies the username and password \\
  \hline
  \textbf{Actor(s)} & Admin \\
  \hline
  \textbf{Pre-conditions} & The admin is logged in, and is on the admin homepage\\
  \hline
  \textbf{Basic Flow} & \begin{itemize}
    \item[] \textbf{1} The admin chooses to create a new instructor account
    \item[] \textbf{2} The system prompts the admin to specify the instructor name, username, and password
    \item[] \textbf{3} The admin fills in the name, username and password, and submits
    \item[] \textbf{4} The system creates the account and returns the admin to the admin homepage
  \end{itemize}\\
  \hline
  \textbf{Alternate Flows} & \begin{itemize}
    \item[] \textbf{3A} The user inputs a username that alraedy exists
    \item[] \textbf{3A1} The system displays an error asking the admin to choose another username
  \end{itemize} \\
  \hline
\end{tabular}

\subsubsection{Create Course}

\paragraph{Description} The instructor creates a course with a name and description


\paragraph{Functional Requirements and Use Cases}
\begin{itemize}
  \item \textbf{REQ-5} The system should allow an instructor to create a new course with a name and description
\end{itemize}

\vspace{0.1in}

\begin{tabular}{| p{0.25\linewidth} | p{0.65\linewidth} |}
  \hline
  \textbf{Use Case Name} & Create Course \\
  \hline
  \textbf{Brief Description} & The instructor creates a course \\
  \hline
  \textbf{Actor(s)} & Instructor \\
  \hline
  \textbf{Pre-conditions} & The instructor is logged in, and is currently on the instructor homepage\\
  \hline
  \textbf{Basic Flow} & \begin{itemize}
    \item[] \textbf{1} The instructor chooses to create a new course
    \item[] \textbf{2} The system prompts the instructor to fill in the new course name and description
    \item[] \textbf{3} The instructor fills in the details and submits
    \item[] \textbf{4} The system creates the course, and displays the course's page
  \end{itemize}\\
  \hline
  \textbf{Alternate Flows} & \begin{itemize}
    \item[] \textbf{3A} The instructor already has a course with the given coursename
    \item[] \textbf{3A1} The system displays an error asking the instructor to choose another course name
  \end{itemize} \\
  \hline
\end{tabular}

\subsubsection{Create Student Account}

\subsubsection{View Assignments}

\subsubsection{Submitting and Reviewing Assignments}

\subsubsection{Instructor Viewing Submitted Assignments}

\subsubsection{Submission Feedback}


\newpage
\appendix


\section{Glossary}

\paragraph{Test Case} A test case is an execution instance that tests submission code for expected results. Test cases consist of two parts: an input and output. The input is what is fed into the code, and the output is what the code should output. The actual output is compared with the expected output to pass or fail the test.


\newpage
\section{Analysis Models}
\subsection{Use Case Diagram}
\includegraphics[height=8in,width=\linewidth,keepaspectratio]{Use Case Diagram.png}

\newpage
\section{To Be Determined List}


\end{document}
